
\pagestyle{empty}\thispagestyle{empty}
\vspace*{\fill}
\vspace*{5ex}
\begin{center}
	\fontsize{40}{40}\selectfont
	\textsc{deepmath}
	
	\vspace*{1ex}
	\textsc{\fontsize{24}{24}\selectfont 
	mathématiques (simples) \\
	des réseaux de neurones \\[-18pt]
	(pas trop compliqués)
	}
	
	\vspace*{2ex}
	
	%\fontsize{32}{32}\selectfont
	\Large
	\textsc{arnaud bodin \ \& \ françois recher}

\end{center}
\vfill
\begin{center}
	\Large
	\textsc{algorithmes \  et \  mathématiques}
\end{center}
\begin{center}
	\LogoExoSept{2}
\end{center}

\clearemptydoublepage
%\clearpage

\thispagestyle{empty}

\vspace*{\fill}
\section*{Mathématiques des réseaux de neurones}

%---------------------------
{\large\textbf{Introduction}}

Ce livre comporte trois parties avec pour chacune un côté \og{} mathématique\fg{} et un côté \og{} réseau de neurones\fg{} :
\begin{itemize}
  \item analyse et réseaux de neurones,
  \item algèbre et convolution,
  \item probabilités appliquées aux réseaux.
\end{itemize}

Le but de la première partie est de comprendre les mathématiques liées aux réseaux de neurones et le calcul des poids par rétropropagation. La seconde est consacrée à la convolution qui est une opération mathématique simple pour extraire des caractéristiques d'une image et permet d'obtenir des réseaux de neurones performants. Enfin, la dernière partie est un court chapitre sur les probabilités rencontrées dans nos réseaux.

\medskip

Nous limitons les mathématiques présentées au niveau de la première année d'études supérieures, ce qui permet de comprendre les calculs de la rétropropagation.
Ce livre explique comment utiliser des réseaux de neurones simples (à l'aide de \tensorflow/\keras). À l'issue de sa lecture, vous saurez programmer un réseau qui distingue un chat d'un chien ! Cependant ce n'est un manuel ni d'ingénierie ni d'optimisation informatique. 
Le lecteur est donc supposé être familier avec les mathématiques du niveau du lycée  (dérivée, étude de fonctions\ldots) et avec les notions de base de la programmation avec \Python{}.

\medskip

Selon votre profil vous pouvez suivre différents parcours de lecture :
\begin{itemize}
  \item le \emph{novice} étudiera le livre dans l'ordre, les chapitres alternant théorie et pratique. Le danger est de se perdre dans les premiers chapitres préparatoires et de ne pas arriver jusqu'au c\oe ur du livre.

  \item le \emph{curieux} picorera les chapitres selon ses intérêts. Nous vous conseillons alors d'attaquer directement par le chapitre \og{}Réseau de neurones\fg{} puis de revenir en arrière pour revoir les notions nécessaires.
    
  \item le \emph{matheux} qui maîtrise déjà les fonctions de plusieurs variables pourra commencer par approfondir ses connaissances de \Python{} avec \numpy{} et \matplotlib{} et pourra ensuite aborder \tensorflow/\keras{} sans douleurs. Il faudra cependant comprendre la \og{}dérivation automatique\fg{}.
  
  \item l'\emph{informaticien} aura peut-être besoin de revoir les notions de mathématiques, y compris les fonctions d'une variable qui fournissent un socle solide, avant d'attaquer les fonctions de deux variables ou plus.
\end{itemize}


\bigskip

Ce livre est encore en version \og{}beta\fg{} et nous avons besoin de votre aide :
\begin{itemize}
  \item merci de nous signaler toutes les fautes (de calcul, de programmation, d'orthographe),
  \item n'hésitez pas nous à dire si certaines explications ne sont pas claires,
  \item vous pouvez nous proposer des ajouts ou des améliorations.
%  \item nous aurions aussi besoin d'un meilleur titre pour le livre !
\end{itemize}


  

\bigskip
\vspace*{\fill}
\begin{center}
L'intégralité des codes \Python{} ainsi que tous les fichiers sources sont sur la page \emph{GitHub} d'Exo7 :\\
\href{https://github.com/exo7math/}{\og{}GitHub : Exo7\fg{}}.
\end{center}



%\vspace*{\fill}


%\newpage
\cleardoublepage
\thispagestyle{empty}
\addtocontents{toc}{\protect\setcounter{tocdepth}{0}}
\tableofcontents


